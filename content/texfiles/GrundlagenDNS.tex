\hypertarget{domain-name-system}{%
\subsection{Domain Name System}\label{domain-name-system}}

Das Internet ist ohne die Technologie des \emph{Domain Name Systems}, im
folgenden mit DNS abgekürzt, kaum noch vorstellbar. Über diesen Service
ist es möglich für den Menschen einfach lesbare, alphabetische Namen,
statt mehrstellige Nummern zu nutzen, um Ressourcen in einem Netzwerk
aufzurufen und auf diese zuzugreifen. Diese alphanumerischen Namen
werden ebenfalls als \emph{Domain Names} bezeichnet. Das Routing in
einem großen Netzwerk wie dem Internet wäre jedoch schwer über solche
Namen zu realisieren, stattdessen werden hierzu numerische IP-Adressen
verwendet. Um trotzdem die besser nutzbaren alphabetischen Ausdrücke
nutzen zu können ist also ein Service zur Übersetzung dieser in
IP-Adressen gefordert. Dieses Technologie bezeichnet man als
\emph{Domain Name System}.

Dies bedeutet, jedes Mal wenn der Domain Name einer Internetseite
eingegeben wird, muss diese zunächst in das IP-System übersetzt werden.
Die Auflösung des Domain Namens kann jedoch nicht durch die Übersetzung
der Zeichen erfolgen, sondern die Funktionsweise ähnelt mehr der eines
Telefonbuchs. Auf einem DNS Server ist eine Liste hinterlegt. Jede Tupel
in dieser Liste besteht aus zwei Werten. Dem Domain Namen und der
dazugehörigen IP-Adresse. Wird nun über einen sogenannten DNS-Lookup ein
Domain Name überliefert, kann nun als Antwort die passende IP-Adresse
zurückgegeben werden, über welche die erfragte Ressource erreicht werden
kann.
